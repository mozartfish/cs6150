\documentclass[addpoints]{exam}
\usepackage{url}
\usepackage{amsmath,amsthm,enumitem}
\usepackage{graphicx}
\usepackage{algorithm, algorithmicx, algpseudocode}
%\input myfonts

\renewcommand{\gcd}{\textsc{Gcd}}
\newtheorem*{claim}{Claim}

\title{CS 6150: HW0 -- Introduction and background}
\date{Submission date: Saturday, August 28, 2021, 11:59 PM}
\begin{document}
\maketitle
\begin{center}
\fbox{\fbox{\parbox{5.5in}{\centering
This assignment has \numquestions\ questions, for a total of \numpoints\
points. % and \numbonuspoints\ bonus points.  
Unless otherwise specified, complete and reasoned arguments will be expected for all answers. }}}
\end{center}

\qformat{Question \thequestion: \thequestiontitle\dotfill \textbf{[\totalpoints]}}
\pointname{}
\bonuspointname{}
\pointformat{[\bfseries\thepoints]}


\begin{center}
  \gradetable
\end{center}
\newpage
\begin{questions}

\titledquestion{Big oh and running times}

\begin{parts}
\part[4] Write down the following functions in big-oh notation:
	\begin{enumerate}
	\item $f(n) = n^2 + 5n + 20$.
	\item $g(n) = \frac{1}{n^2} + \frac{2}{n}$.
	\end{enumerate}
\part[6] Consider the following algorithm to compute the GCD of two positive integers $a, b$. 
Suppose $a, b$ are numbers that are both at most $n$. Give a bound on the running time of $\gcd(a, b)$. (You need to give a formal proof for your claim.)
\begin{algorithm}[!h]
	\caption{$\gcd(a, b)$}
	\begin{algorithmic}
		\State if ($a < b$) return $\gcd(b, a)$;
		\State if ($b = 0$) return $a$;
		\State return $\gcd(b, a\%b)$;  (Recall: $a\%b$ is the remainder when $a$ is divided by $b$)
	\end{algorithmic}
\end{algorithm}
\end{parts}

\titledquestion{Square vs. Multiply}[5]
Suppose I tell you that there is an algorithm that can square any $n$ digit number in time $O(n \log n)$, for all $n\ge 1$. Then, prove that there is an algorithm that can find the product of {\em any two} $n$ digit numbers in time $O(n \log n)$.  [{\em Hint:} think of using the squaring algorithm as a subroutine to find the product.]

\titledquestion{Graph basics}[8]
Let $G$ be a {\em simple}\footnote{I.e., there are no self loops or multiple edges between any pair of vertices.} undirected graph. Prove that there are at least two vertices that have the same degree. 

\titledquestion{Background: Probability}
\begin{parts}
\part[3] Suppose we toss a fair coin $k$ times. What is the probability that we see heads precisely once? 
\part[4] Suppose we have $k$ different boxes, and suppose that every box is colored uniformly at random with one of $k$ colors (independently of the other boxes). What is the probability that all the boxes get distinct colors?
\part[5] Suppose we repeatedly throw a fair die (with 6 faces). What is the expected number of throws needed to see a `1'? How many throws are needed to ensure that a `1' is seen with probability $> 99/100$?
\end{parts}

\titledquestion{Tossing coins}
Suppose we have two coins, one of which is {\em fair} (i.e. prob[heads] = prob[tails] = $1/2$), and another of which is slightly biased. More specifically, the second coin has prob[heads] = 0.51. Suppose we toss the coins $N$ times, and let $H_1$ and $H_2$ be the number of heads observed (respectively).
\begin{parts}
\part[3] Intuitively, how large must $N$ be, so that we have $H_2 > H_1$ with ``reasonable certainty''?
\\ 
\part[2] Suppose we pick $N = 25$. What is the expected value of $H_2 - H_1$?
\part[2] Can you use this to conclude that the probability of the event $(H_2 - H_1 \ge 1)$ is small? [It's OK if you cannot answer this part of the problem.] 
\end{parts}


\titledquestion{Array Sums}[8]
Given an array $A[1 \dots n]$ of integers, find if there exist indices $i, j, k$ such that $A[i]+A[j]+A[k] = 0$.  Can you find an algorithm with running time $o(n^3)$? [NOTE: this is the little-oh notation, i.e., the algorithm should run in time $< cn^3$, for any constant $c$, as $n \rightarrow \infty$.] [{\em Hint:} aim for an algorithm with running time $O(n^2 \log n)$.]

\end{questions}
\end{document}
