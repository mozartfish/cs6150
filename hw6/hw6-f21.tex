\documentclass[addpoints]{exam}
\usepackage{url}
\usepackage{amsmath, amsthm, enumitem, amssymb}
\usepackage{graphicx}
%\input myfonts

\newtheorem*{claim}{Claim}
\title{CS 6150: HW 6 -- Optimization formulations, review}
\date{Submission date: Tuesday, December 14, 2021, 11:59 PM}
\begin{document}
\maketitle
\begin{center}
\fbox{\fbox{\parbox{5.5in}{\centering
This assignment has \numquestions\ questions, for a total of \numpoints\
points. % and \numbonuspoints\ bonus points.  
Unless otherwise specified, complete and reasoned arguments will be expected for all answers. }}}
\end{center}

\qformat{Question \thequestion: \thequestiontitle\dotfill \textbf{[\totalpoints]}}
\pointname{}
\bonuspointname{}
\pointformat{[\bfseries\thepoints]}


\begin{center}
  \gradetable
\end{center}
\newpage

\newcommand{\R}{\mathbb{R}}
\newcommand{\E}{\mathbb{E}}
\newcommand{\bu}{\mathbf{u}}
\newcommand{\iprod}[1]{\langle #1 \rangle}
\newcommand{\norm}[1]{\lVert #1 \rVert}

\begin{questions}

\titledquestion{Understanding relax-and-round}
In this question, you will implement the relax and round paradigm using the example of the Set Cover (or hiring) problem. Suppose we have $n$ people (i.e., sets) and $m$ skills that we wish to cover. Let us create an instance with $n=m=500$. Let $d = 25$ be the size of each skill set.
\begin{parts}
\part[3] Write code that generates a random instance of set cover, where for each person $i$, the skill set $S_i$ is a random subset of the $m$ skills, with $|S_i|=d$. 
\part[3] Write down the integer linear program using variables $x_i$ that indicate if person $i$ is chosen/hired (abstractly, as we did in class). Then write down the linear programming relaxation. Which one has the lower optimum objective value?
\part[6] For the instance you created in part (a), solve the linear program from part (b) using an LP solver of your choice, and output the fractional solution.
\part[4] Round the fractional solution using randomized rounding, i.e., hire person $i$ with probability $\min (1, t x_i)$. Try $t=1, 2, 4, 8$, and in each case, report the (a) total number of people hired, and (b) number of ``uncovered'' skills (i.e., skills for which none of the people possessing the skill were hired). 
\end{parts}


\titledquestion{The power of two choices}

As I mentioned in class, one of the classic applications of random hashing is ``on-the-fly'' load balancing. In this problem, we will empirically see how ``balanced'' such an assignment is, and how to make it more balanced.

In what follows, set $N = 10^7$, i.e., 10 million. Suppose we have $N$ servers, and $N$ service requests arrive sequentially. 

\begin{parts}
\part[4] When a request arrives, suppose we generate a random index $r$ between 1 and $N$ and send the request to server $r$ (and we do this independently for each request). Plot a histogram showing the distribution of the ``loads'' of the servers in the end. I.e., show how many servers have load 0, load 1, and so on. [The load is defined as the number of requests routed to that server.]
\part[6] Now, suppose we do something slightly smarter: when a request arrives, we generate {\em two} random indices $r_1$ and $r_2$ between $1$ and $N$, query to find the {\em current load} on the servers $r_1$ and $r_2$, and assign the request to the server with the {\em lesser} load (breaking ties arbitrarily). With this allocation, plot the histogram showing the load distribution, as above.
\end{parts}

\titledquestion{Optimal packaging}[10]

With the holiday season around the corner, company Bezo wants to minimize the number of shipping boxes. Let us consider the one dimensional version of the problem: suppose a customer orders $n$ items of lengths $a_1, a_2, \dots, a_n$ respectively, and suppose $0 < a_i \le 1$. The goal is to place them into boxes of length 1 such that the total {\bf number of boxes} is minimized.  

It turns out that this is a rather difficult problem. But now, suppose that there are only $r$ {\em distinct} values that the lengths could take. In other words, suppose that there is some set $L = \{s_1, \dots, s_r\}$ such that every $a_i \in L$. Let us think of $r$ is as a small constant.  Devise an algorithm that runs in time $O(n^r)$, and computes the optimal number of boxes.

[{\em Hint:} first find all the possible ``configurations'' that can fit in a single box. Then use dynamic programming.] 

\titledquestion{Non-negativity in Markov}[4]
Markov's inequality states that for a non-negative random variable $X$, we have $\Pr[ X > t \cdot \E[X] ] \le 1/t$, for any $t \ge 1$. 

The point of this exercise is to show that the non-negativity is important. Give an example of a random variable (that takes negative values), for which (a) $\E[X] = 1$, and (b) $\Pr[ X > 5] \ge 0.9$.

\titledquestion{Distributed independent set}
A fundamental problem in distributed algorithms (used in P2P networks, distributed coloring, etc.) is the following:  we are given an undirected graph with $n$ vertices where each vertex corresponds to an `agent'. The goal is to find a large independent set (a set of vertices with no edges between them) in a distributed manner. It turns out that this problem has a nice solution when the degree of each vertex is $\le d$, for some known parameter $d$.

Consider the following algorithm. (1) Every vertex becomes {\em active} with probability $\frac{1}{2d}$.  (2) Every active vertex queries its neighbors, and if any vertex in the neighborhood is also active, it becomes inactive. (Step (2) is done in parallel; thus if $i$ and $j$ are neighbors and they were both activated in step (1), they both become inactive.) (3) The set of active vertices in the end is output as the independent set.

\begin{parts}
\part[2] Let $X$ be the random variable that is the number of vertices activated in step (1). Find $\E[X]$. 
\part[3] Let $Y$ be the random variable that is the number of edges $\{i,j\}$ {\em both of whose end points} are activated in step (1). Find $\E[Y]$ (in terms of $m$, the total number of edges in the graph).
\part[5] Prove that the size of the independent set output in (3) is at least $X - 2Y$, and thus show that the expectation of this quantity is $\ge n/4d$.
\end{parts}

\end{questions}
\end{document}